\chapter{\uppercase{Instalación y configuración de la red}}\label{chap:ins.conf.red}

\newpage

\section{Consultar el nombre de las interfaces de red}
Antes de instalar y configurar la red temporalmente, se requiere conocer el nombre de las interfaces, para esto ejecutamos el comando:\\
\begin{minted}[breaklines,style=colorful,]{shell-session}
# ip addr
\end{minted}
\textattachfile[mimetype=text/plain, description= Archivo comando.txt abierto, color=0 0 0.7, print=false]{commands/con.nom.int.red}{comando.txt}\\[1em]
El nombre que se requiere, es algo como enp2s0 ó eth0 (ethernet), wlp7s5 ó wlan0 (wifi).

\section{Configuración temporal de las interfaces de red}
Permite realizar una conexion temporal a internet para descargar e instalar los paquetes esenciales. Para esto modificaremos los archivos de configuracion de las interfaces:\\
Interfaz ethernet con DHCP:
\begin{minted}[breaklines,style=colorful,]{shell-session}
# nano /etc/network/interfaces

source /etc/network/interfaces.d/*

auto lo
iface lo inet loopback

auto [NOMBRE_INTERFACE]
allow-hotplug [NOMBRE_INTERFACE]
iface [NOMBRE_INTERFACE] inet dhcp
\end{minted}
\textattachfile[mimetype=text/plain, description= Archivo comando.txt abierto, color=0 0 0.7, print=false]{commands/con.tem.int.red.int.eth.dhc}{comando.txt}\\[1em]
Interfaz wifi con DHCP:
\begin{minted}[breaklines,style=colorful,]{shell-session}
# nano /etc/network/interfaces
 
source /etc/network/interfaces.d/*

auto lo
iface lo inet loopback

auto [NOMBRE_INTERFACE]
allow-hotplug [NOMBRE_INTERFACE]
iface [NOMBRE_INTERFACE] inet dhcp
	wpa-ssid [SSID]
	wpa-psk [CLAVE_SSID]
\end{minted}
\textattachfile[mimetype=text/plain, description= Archivo comando.txt abierto, color=0 0 0.7, print=false]{commands/con.tem.int.red.int.wif.dhc}{comando.txt}\\[1em]
Interfaz ethernet con IP estatica:
\begin{minted}[breaklines,style=colorful,]{shell-session}
# nano /etc/network/interfaces
 
source /etc/network/interfaces.d/*

auto lo
iface lo inet loopback

auto [NOMBRE_INTERFACE]
allow-hotplug [NOMBRE_INTERFACE]
iface [NOMBRE_INTERFACE] inet static
	address 192.168.0.11
	netmask 255.255.255.0
	network 192.168.0.0
	gateway 192.168.0.1
	broadcast 192.168.0.255
	dns-nameservers 8.8.8.8 8.8.4.4
\end{minted}
\textattachfile[mimetype=text/plain, description= Archivo comando.txt abierto, color=0 0 0.7, print=false]{commands/con.tem.int.red.int.eth.est}{comando.txt}\\[1em]

\section{Agregar resolvers de DNS \emph{(*Opcional)}}
Puede suceder que no reconozca el parametro \textbf{dns-nameservers} en la configuración de las interfaces y sea necesario agregarlo tambien en los resolvers.
\begin{minted}[breaklines,style=colorful,]{shell-session}
# nano /etc/resolv.conf

nameserver 8.8.8.8
nameserver 8.8.4.4
\end{minted}
\textattachfile[mimetype=text/plain, description= Archivo comando.txt abierto, color=0 0 0.7, print=false]{commands/agr.res.dns}{comando.txt}\\[1em]

\section{Gestionando el servicio de gestion de la red}
Para aplicar los cambios anteriores se requiere reiniciar el servicio de gestion de la red. Utilizamos el primer restart de la lista siguiente:
\begin{minted}[breaklines,style=colorful,]{shell-session}
# /etc/init.d/networking stop
# /etc/init.d/networking status
# /etc/init.d/networking start
# /etc/init.d/networking restart
# service network-manager stop
# service network-manager status
# service network-manager start
# service network-manager restart
\end{minted}
\textattachfile[mimetype=text/plain, description= Archivo comando.txt abierto, color=0 0 0.7, print=false]{commands/ges.ser.ges.red}{comando.txt}\\[1em]

\section{¿Estamos conectados a internet?}
Para responder a esta pregunta, utilizamos el comando ping.
\begin{minted}[breaklines,style=colorful,]{shell-session}
# ping google.com
\end{minted}
\textattachfile[mimetype=text/plain, description= Archivo comando.txt abierto, color=0 0 0.7, print=false]{commands/est.con.red}{comando.txt}\\[1em]
Una vez conectados, comenzamos a descargar los paquetes esenciales.\\
Si aun se tiene problemas con la red, es aconsejable utilizar una conexion ethernet.\\

\section{Instalar herramientas de red}
Este paquete contiene importantes herramientas para gestionar la red como: arp, ifconfig, netstat, rarp, nameif y route. Adicionalmente contiene utilidades relacionada como plipconfig, slattach, mii-tool, iptunnel e ipmaddr.
\begin{minted}[breaklines,style=colorful,]{shell-session}
# apt install net-tools
\end{minted}
\textattachfile[mimetype=text/plain, description= Archivo comando.txt abierto, color=0 0 0.7, print=false]{commands/con.nom.int.red}{comando.txt}\\[1em]

\section{Problema con el driver de la tarjeta wifi}
Para solucionar estos problemas, se requiere conexion a internet por cable ethernet.\\
Para conocer si el sistema operativo reconoce la tarjeta wifi, utilizamos el comando lspci, el cual permite desplegar la informacion acerca de los buses PCI.
\begin{minted}[breaklines,style=colorful,]{shell-session}
# lspci -nn | grep -i network
\end{minted}
\textattachfile[mimetype=text/plain, description= Archivo comando.txt abierto, color=0 0 0.7, print=false]{commands/con.nom.int.red}{comando.txt}\\[1em]
Puede que el driver \textbf{iwlwifi} o \textbf{firmware-iwlwifi} se haya instalado (prueba utilizando el comando \textbf{modinfo iwlwifi} o el lspci, si no se reconoce entonces no esta instalado) y es necesario reinstalarlo o reconfigurarlo.\\
Comandos utilizados: modprobe (Agrega un modulo de kernel), rmmod (Remueve un modulo del kernel), iwlwifi (Driver con soporte para varios adaptadores wifi intel).
\begin{minted}[breaklines,style=colorful,]{shell-session}
# ifconfig [NOMBRE_INTERFACE] down
# rmmod -f iwlwifi
# modprobe iwlwifi 11n_disable=1
# ifconfig [NOMBRE_INTERFACE] up
\end{minted}
\textattachfile[mimetype=text/plain, description= Archivo comando.txt abierto, color=0 0 0.7, print=false]{commands/con.nom.int.red}{comando.txt}\\[1em]
Algunas veces es necesario apagar y no reiniciar, para que los cambios surgan efecto (Probablemente se almacenen algunas configuraciones en la memoria RAM, despues de el reinicio).\\

\section{Problema con la MAC aleatoria del NetworkManager}
El NetworkManager agrego la direccion MAC aleatoria para incrementar la privacidad, pero esto puede generar errores de conexion, prueba deshabilitando este parametro en su configuración.
\begin{minted}[breaklines,style=colorful,]{shell-session}
# nano /etc/NetworkManager/NetworkManager.conf

[main]
plugins=ifupdown,keyfile

[ifupdown]
managed=false

[device]
wifi.scan-rand-mac-address=no
\end{minted}
\textattachfile[mimetype=text/plain, description= Archivo comando.txt abierto, color=0 0 0.7, print=false]{commands/con.nom.int.red}{comando.txt}\\[1em]
Algunas veces es necesario apagar y no reiniciar, para que los cambios surgan efecto (Probablemente se almacenen algunas configuraciones en la memoria RAM, despues de el reinicio).\\

\section{Reconfiguración de la red}
Esta reconfiguración la aplicaremos cuando se culmine con el \autoref{chap:ins.sof.ese}\\
Los cambios aplicados en el archivo \textbf{/etc/network/interfaces}
\begin{minted}[breaklines,style=colorful,]{shell-session}
# nano /etc/network/interfaces
 
source /etc/network/interfaces.d/*

auto lo
iface lo inet loopback

# auto [NOMBRE_INTERFACE]
# allow-hotplug [NOMBRE_INTERFACE]
# iface [NOMBRE_INTERFACE] inet dhcp
#   wpa-ssid [SSID]
#   wpa-psk [CLAVE_SSID]
\end{minted}
\textattachfile[mimetype=text/plain, description= Archivo comando.txt abierto, color=0 0 0.7, print=false]{commands/con.nom.int.red}{comando.txt}\\[1em]

\begin{minted}[breaklines,style=colorful,]{shell-session}
# /etc/init.d/networking restart
\end{minted}
\textattachfile[mimetype=text/plain, description= Archivo comando.txt abierto, color=0 0 0.7, print=false]{commands/con.nom.int.red}{comando.txt}\\[1em]

